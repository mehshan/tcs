\documentclass[11pt]{article}

\usepackage{fullpage}
\usepackage{graphicx}
\usepackage{amsmath}
\usepackage{amssymb}
\usepackage{amsthm}
\usepackage{fancyvrb}

\parindent0in
\pagestyle{plain}
\thispagestyle{plain}

\newcommand{\myname}{Mehshan Mustafa}
\newcommand{\dated}{\today}

\newenvironment{theorem}[2][Theorem]{\begin{trivlist}
\item[\hskip \labelsep {\bfseries #1}\hskip \labelsep {\bfseries #2.}]}{\end{trivlist}}
\newenvironment{lemma}[2][Lemma]{\begin{trivlist}
\item[\hskip \labelsep {\bfseries #1}\hskip \labelsep {\bfseries #2.}]}{\end{trivlist}}
\newenvironment{exercise}[2][Exercise]{\begin{trivlist}
\item[\hskip \labelsep {\bfseries #1}\hskip \labelsep {\bfseries #2.}]}{\end{trivlist}}
\newenvironment{problem}[2][Problem]{\begin{trivlist}
\item[\hskip \labelsep {\bfseries #1}\hskip \labelsep {\bfseries #2.}]}{\end{trivlist}}
\newenvironment{question}[2][Question]{\begin{trivlist}
\item[\hskip \labelsep {\bfseries #1}\hskip \labelsep {\bfseries #2.}]}{\end{trivlist}}
\newenvironment{corollary}[2][Corollary]{\begin{trivlist}
\item[\hskip \labelsep {\bfseries #1}\hskip \labelsep {\bfseries #2.}]}{\end{trivlist}}
\newenvironment{solution}{\begin{proof}[Solution]}{\end{proof}}
\newenvironment{idea}[2][Proof Idea.]{\textit{#1} #2}

\begin{document}

\textbf{Introduction to the Theory of
Computation}\hfill\textbf{\myname}\\[0.01in]
\textbf{Chapter 1: Reqular Languages}\hfill\textbf{\dated}\\
\smallskip\hrule\bigskip

\begin{problem}{1.66}
A homomorphism is a function $f : \Sigma \longrightarrow \Gamma^{*}$ from one alphabet to strings over another alphabet. We can extend $f$ to operate on strings by defining $f(w) = f(w_{1})f(w_{2}) \cdots f(w_{n})$, where $w = w_{1}w_{2} \cdots w_{n}$ and  each $w_{i} \in \Sigma$. We further
extend $f$ to operate on languages by defining $f(A) = \{f(w) \ | \ w \in A\}$, for any language A.
\end{problem}

\begin{problem}[Part]{a}
Show, by giving a formal construction, that the class of regular languages
is closed under homomorphism. In other words, given a DFA $M$ that recognizes $B$ and a homomorphism $f$, construct a finite automaton $M'$ that recognizes $f(B)$. Consider the machine $M'$ that you constructed. Is it a DFA in every case?
\end{problem}

\begin{proof}
Solution Replace this text with the details of your proof or solution.
\end{proof}

\begin{problem}[Part]{b}
Show, by giving an example, that the class of non-regular languages is not
closed under homomorphism.
\end{problem}

\begin{proof}
Solution Replace this text with the details of your proof or solution.
\end{proof}

\end{document}
\documentclass[11pt]{article}

\usepackage{fullpage}
\usepackage{graphicx}
\usepackage{amsmath}
\usepackage{amssymb}
\usepackage{amsthm}
\usepackage{fancyvrb}

\parindent0in
\pagestyle{plain}
\thispagestyle{plain}

\newcommand{\myname}{Mehshan Mustafa}
\newcommand{\dated}{\today}

\newenvironment{theorem}[2][Theorem]{\begin{trivlist}
\item[\hskip \labelsep {\bfseries #1}\hskip \labelsep {\bfseries #2.}]}{\end{trivlist}}
\newenvironment{lemma}[2][Lemma]{\begin{trivlist}
\item[\hskip \labelsep {\bfseries #1}\hskip \labelsep {\bfseries #2.}]}{\end{trivlist}}
\newenvironment{exercise}[2][Exercise]{\begin{trivlist}
\item[\hskip \labelsep {\bfseries #1}\hskip \labelsep {\bfseries #2.}]}{\end{trivlist}}
\newenvironment{problem}[2][Problem]{\begin{trivlist}
\item[\hskip \labelsep {\bfseries #1}\hskip \labelsep {\bfseries #2.}]}{\end{trivlist}}
\newenvironment{question}[2][Question]{\begin{trivlist}
\item[\hskip \labelsep {\bfseries #1}\hskip \labelsep {\bfseries #2.}]}{\end{trivlist}}
\newenvironment{corollary}[2][Corollary]{\begin{trivlist}
\item[\hskip \labelsep {\bfseries #1}\hskip \labelsep {\bfseries #2.}]}{\end{trivlist}}
\newenvironment{solution}{\begin{proof}[Solution]}{\end{proof}}
\newenvironment{idea}[2][Proof Idea.]{\textit{#1} #2}

\begin{document}
\textbf{Introduction to the Theory of
Computation}\hfill\textbf{\myname}\\[0.01in]
\textbf{Chapter 1: Reqular Languages}\hfill\textbf{\dated}\\
\smallskip\hrule\bigskip

\begin{problem}{1.38}
An all-NFA $M$ is a 5-tuple $(Q, \Sigma, \delta, q_{0}, F)$ that accepts $x \in \Sigma^{*}$ if every posible state that $M$ could be in after reading input $x$ is a state from $F$. Note, in contrast, that an ordinary NFA accepts a string if some state among these possible states is an accept state. Prove that all-NFAs recognize the class of regular languages.
\end{problem}

\begin{proof}
The proof is in two parts. In the first part, show that for any regular language $A$ and an NFA $N$ that recognizes it, there exists an all-NFA $M$ that reocognizes the complement of $A$. In the second part, use the result of the first part to show that for every regular language there exists an all-NFA that recognizes it.
\\
\\
\textbf{Part 1.} Let $A$ be any regular language, and let $N$ be an NFA that recognizes A. Use the construction give in the Theorem 1.39 to construct an equivalent DFA $D$ for $N$ that also recognizes $A$. Swap the accept and non-accept states of $D$ to get a new DFA $D'$ that accepts the complement of $A$\footnote{Exercise 1.14 a.}.
\\
\\
Let $D' = (Q', \Sigma, \delta', q_{0}', F')$. Construct the all-NFA $M = (Q, \Sigma, \delta, q_{0}, F)$ by reversing the construction of the Theorem 1.39.
\begin{enumerate}
\item $Q = Q'$
\item $q_{0} = \{q_{0}'\}$
\end{enumerate}
\textbf{Part 2.}
\end{proof}
\end{document}
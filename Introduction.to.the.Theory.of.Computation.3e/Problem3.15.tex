\documentclass[11pt]{article}


\usepackage{fullpage}
\usepackage{graphicx}
\usepackage{amsmath}
\usepackage{amssymb}
\usepackage{amsthm}
\usepackage{fancyvrb}

\newcommand{\myname}{Mehshan Mustafa}

\newenvironment{theorem}[2][Theorem]{\begin{trivlist}
\item[\hskip \labelsep {\bfseries #1}\hskip \labelsep {\bfseries #2.}]}{\end{trivlist}}
\newenvironment{lemma}[2][Lemma]{\begin{trivlist}
\item[\hskip \labelsep {\bfseries #1}\hskip \labelsep {\bfseries #2.}]}{\end{trivlist}}
\newenvironment{exercise}[2][Exercise]{\begin{trivlist}
\item[\hskip \labelsep {\bfseries #1}\hskip \labelsep {\bfseries #2.}]}{\end{trivlist}}
\newenvironment{problem}[2][Problem]{\begin{trivlist}
\item[\hskip \labelsep {\bfseries #1}\hskip \labelsep {\bfseries #2.}]}{\end{trivlist}}
\newenvironment{question}[2][Question]{\begin{trivlist}
\item[\hskip \labelsep {\bfseries #1}\hskip \labelsep {\bfseries #2.}]}{\end{trivlist}}
\newenvironment{corollary}[2][Corollary]{\begin{trivlist}
\item[\hskip \labelsep {\bfseries #1}\hskip \labelsep {\bfseries #2.}]}{\end{trivlist}}
\newenvironment{solution}{\begin{proof}[Solution]}{\end{proof}}
\newenvironment{idea}[2][Proof Idea.]{\textit{#1} #2}



\parindent0in
\pagestyle{plain}
\thispagestyle{plain}


\usepackage{csquotes}

\newcommand{\dated}{\today}
\newcommand{\token}[1]{\langle \text{#1} \rangle}

\begin{document}

\textbf{Introduction to the Theory of
Computation}\hfill\textbf{\myname}\\[0.01in]
\textbf{Chapter 3: The Church-Turing Thesis}\hfill\textbf{\dated}\\
\smallskip\hrule\bigskip

\begin{problem}{3.15}
Show that the collection of decidable languages is closed under the operation of.
\end{problem}

\begin{problem}[Part]{b}
concatenation.
\end{problem}

For any two decidable languages $L_1$ and $L_2$, let $M_1$ and $M_2$ be the \textbf{TM}s that decide them. For a string $w = w_1w_2w_3 \cdots w_n$, define split $W = (P, \ S)$, where $P, S \in \Sigma^*$, and $w = PS$. In other words, the strings $P$ and $S$ are prefix and suffix of $w$, such that their concatenation is equivalent to $w$. We construct a \textbf{TM} $M^{'}$ that decides the concatenation of $L_1$ and $L_2$: \\
\\
\textquotedblleft On input $w$:
\begin{enumerate}
\item Calculate the list of all possible splits $W_1 = (P_1, S_1), W_2 = (P_2, S_2), \cdots W_n = (P_n, S_n)$ of $w$.
\item Repeat the following for each $i = 1, 2, 3, \cdots$.
\item \hspace*{0.5cm} Run $M_1$ on $P_i$, and $M_2$ on $S_i$.
\item \hspace*{0.5cm} If in any computation, both $M_1$ and $M_2$ accept, \textit{accept}. \textquotedblright
\end{enumerate}

\begin{problem}[Part]{c}
star.
\end{problem}

For any decidable language $L$, let $M$ be the \textbf{TM} that decides it.

For any non-empty string $w = w_1w_2w_3 \cdots w_n$, define partition $P = (p_1, \  p_2, \ \cdots , \ p_k)$, where $1 \leq k \leq n$, each $p_i \in \Sigma^{*}$, and $w = p_1  p_2 \cdots p_k$. We construct a \textbf{TM} $M^{'}$ that decides $L^{*}$: \\
\\
\textquotedblleft On input $w$:
\begin{enumerate}
\item If $w = \varepsilon$, \textit{accept}.
\item For non-empty $w$, generate list of all possible partitions $P_1, \ P_2, \ P_3, \cdots$.
\item Repeat the following for each $i = 1, 2, 3, \cdots$.
\item \hspace*{0.5cm} Run $M$ on each $p \in P_i$.
\item \hspace*{0.5cm} If in any computation, $M$ accepts all $p \in P_i$, \textit{accept}.\textquotedblright
\end{enumerate}

\begin{problem}[Part]{d}
complementation.
\end{problem}

For any decidable language $L$, let $M$ be the \textbf{TM} that decides it. Construct a \textbf{TM} $M^{'}$ that decides the complement of $L$: \\
\\
\textquotedblleft On input $w$:
\begin{enumerate}
\item Run $M$ on $w$. If it accepts, \textit{reject}. Otherwise, \textit{accept}.\textquotedblright
\end{enumerate}

\begin{problem}[Part]{e}
intersection.
\end{problem}

For any two decidable languages $L_1$ and $L_2$, let $M_1$ and $M_2$ be the \textbf{TM}s that decide them. We construct a \textbf{TM} $M^{'}$ that decides the intersection of $L_1$ and $L_2$: \\
\\
\textquotedblleft On input $w$:
\begin{enumerate}
\item Run $M_1$ on $w$. If it rejects, \textit{reject}.
\item Run $M_2$ on $w$. If it accepts, \textit{accept}. Otherwise, \textit{reject}.\textquotedblright
\end{enumerate}

\end{document}
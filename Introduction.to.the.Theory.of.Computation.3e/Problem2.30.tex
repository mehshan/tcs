\documentclass[11pt]{article}


\usepackage{fullpage}
\usepackage{graphicx}
\usepackage{amsmath}
\usepackage{amssymb}
\usepackage{amsthm}
\usepackage{fancyvrb}

\newcommand{\myname}{Mehshan Mustafa}

\newenvironment{theorem}[2][Theorem]{\begin{trivlist}
\item[\hskip \labelsep {\bfseries #1}\hskip \labelsep {\bfseries #2.}]}{\end{trivlist}}
\newenvironment{lemma}[2][Lemma]{\begin{trivlist}
\item[\hskip \labelsep {\bfseries #1}\hskip \labelsep {\bfseries #2.}]}{\end{trivlist}}
\newenvironment{exercise}[2][Exercise]{\begin{trivlist}
\item[\hskip \labelsep {\bfseries #1}\hskip \labelsep {\bfseries #2.}]}{\end{trivlist}}
\newenvironment{problem}[2][Problem]{\begin{trivlist}
\item[\hskip \labelsep {\bfseries #1}\hskip \labelsep {\bfseries #2.}]}{\end{trivlist}}
\newenvironment{question}[2][Question]{\begin{trivlist}
\item[\hskip \labelsep {\bfseries #1}\hskip \labelsep {\bfseries #2.}]}{\end{trivlist}}
\newenvironment{corollary}[2][Corollary]{\begin{trivlist}
\item[\hskip \labelsep {\bfseries #1}\hskip \labelsep {\bfseries #2.}]}{\end{trivlist}}
\newenvironment{solution}{\begin{proof}[Solution]}{\end{proof}}
\newenvironment{idea}[2][Proof Idea.]{\textit{#1} #2}



\parindent0in
\pagestyle{plain}
\thispagestyle{plain}

\newcommand{\dated}{\today}
\newcommand{\token}[1]{\langle \text{#1} \rangle}

\begin{document}

\textbf{Introduction to the Theory of
Computation}\hfill\textbf{\myname}\\[0.01in]
\textbf{Chapter 2: Context-Free Languages}\hfill\textbf{\dated}\\
\smallskip\hrule\bigskip

\begin{problem}{2.30}
Use the pumping lemma to show that the following languages are not context free.
\end{problem}

\begin{problem}[Part]{a}
$\{0^{n}1^{n}0^{n}1^{n} \ | \ n \geq 0\}$
\end{problem}

\begin{proof}
The proof is by contradiction. Assume the language $A = \{0^{n}1^{n}0^{n}1^{n} \ | \ n \geq 0\}$ is context free. Let $p$ be the pumping length given by the pumping lemma. Choose the string $s = 0^{p}1^{p}0^{p}1^{p}$. Clearly $s$ is a member of $A$ and of length at least $p$. The pumping lemma states that $s$ can be pumped, but we show that it cannot be pumped.

\begin{enumerate}
\item Either $v$ or $y$ is non-empty. Without loss of generality, we only discuss the case when $v$ is non-empty. If $v$ is non-empty, then it may contain only one type of alphabet or both 0's and 1's. In former case, the string $uv^{2}xy^{2}z \notin A$, because it cannot contain equal number of 0's or 1's in each segment. In latter case, the string $uv^{2}xy^{2}z \notin A$ as it cannot contain exactly two segments of 0's and 1's.
\item Both $v$ and $y$ are non-empty. There are three possibilities. First, both $v$ and $y$ contain the same alphabet from the same segment of 0's or 1's. Second, $v$ contains only one type of symbol from a segment and $y$ only contains the other type of symbol from a neighboring segments. Third, $v$ or $y$ contain both 0's and 1's from two neighboring segments. In the first and second cases, the string $uv^{2}xy^{2}z \notin A$, because it cannot contain equal number of 0's and 1's in each segment. In the third case, the string $uv^{2}xy^{2}z \notin A$ as it cannot contain exactly two segments of 0's and 1's.
\end{enumerate}
One of these cases must occur, and both cases result in a contradiction.
\end{proof}

\begin{problem}[Part]{b}
$\{ t_{1} \# t_{2} \# \cdots \# t_{k} \ | \ k \geq 2, \ each \ t_{i} \in \{a, b\}^{*}, \ and \ t_{i} = t_{j} \ for \ some \ i \neq  j \}$
\end{problem}

\begin{proof}
\end{proof}

\end{document}
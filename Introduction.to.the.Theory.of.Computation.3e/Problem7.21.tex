\documentclass[11pt]{article}

\input{Shared.tex}

\parindent0in
\pagestyle{plain}
\thispagestyle{plain}

\usepackage{csquotes}
\usepackage[shortlabels]{enumitem}

\newcommand{\dated}{\today}
\newcommand{\token}[1]{\langle \text{#1} \rangle}

\begin{document}

\textbf{Introduction to the Theory of
Computation}\hfill\textbf{\myname}\\[0.01in]
\textbf{Chapter 7: Time Complexity}\hfill\textbf{\dated}\\
\smallskip\hrule\bigskip

\begin{problem}{7.21}
Let $G$ represent an undirected graph. Also let
\[
SPATH = \{\langle G, a, b, k \rangle \ | \ G \ \text{contains a simple path of length at most } k \text{ from } a \text{ to } b\},
\]
and
\[
LPATH = \{\langle G, a, b, k \rangle \ | \ G \ \text{contains a simple path of length at least } k \text{ from } a \text{ to } b\}.
\]
\end{problem}

\begin{problem}[Part]{a}
Show that $SPATH \in P$.
\end{problem}

\begin{proof}
We show that $SPATH \in P$ by presenting a polynomial time algorithm that decides $SPATH$. A polynomial time algorithm $M$ for $SPATH$ operates as follows. \\

$M =$ \textquotedblleft On input $\langle G, a, b, k \rangle$:
\begin{enumerate}
\item Unmark all nodes.
\item Assign each node a value of $\infty$.
\item Mark $a$, and set it's value 0.
\item Let integer $d = 0$.
\item Repeat until no additional nodes are marked:
\item \hspace*{0.5cm} $d = d + 1$.
\item \hspace*{0.5cm} Scan all edges of $G$. If an edge $(u, v)$ exists between a marked node $u$ and an unmarked \\
 \hspace*{0.5cm} node $v$, then mark $v$ and set value of $v$ to $d$.
\item If $b$ is marked and value of $b$ is at most $k$, then \textit{Accept}. Otherwise \textit{reject}.\textquotedblright
\end{enumerate}
Now we analyze this algorithm to show that it runs in polynomial time. Obviously, stages 1 to 4 and stage 8 are executed only once. e. Stage 7 runs at most m times because each time except the last it marks an additional node in G, where $m$ is the number of nodes in $G$. Stage 6 also runs at most m times. Thus, the total number of stages used is at most $1 + 1 + m + m$, giving a polynomial in the size of G. \\

Stages 1 to 4 along with stages 6 and 8 are easily implemented in polynomial time on any reasonable deterministic model. Stage 7 involves a scan of the input and a test of
whether certain nodes are marked and an update of node values, which also is easily implemented in polynomial time. Hence $M$ is a polynomial time algorithm for $SPATH$.
\end{proof}

\begin{problem}[Part]{b}
Show that $LPATH$ is NP-complete.
\end{problem}

\begin{proof}
\end{proof}

\end{document}
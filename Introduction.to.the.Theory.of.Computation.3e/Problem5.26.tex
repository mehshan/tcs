\documentclass[11pt]{article}


\usepackage{fullpage}
\usepackage{graphicx}
\usepackage{amsmath}
\usepackage{amssymb}
\usepackage{amsthm}
\usepackage{fancyvrb}

\newcommand{\myname}{Mehshan Mustafa}

\newenvironment{theorem}[2][Theorem]{\begin{trivlist}
\item[\hskip \labelsep {\bfseries #1}\hskip \labelsep {\bfseries #2.}]}{\end{trivlist}}
\newenvironment{lemma}[2][Lemma]{\begin{trivlist}
\item[\hskip \labelsep {\bfseries #1}\hskip \labelsep {\bfseries #2.}]}{\end{trivlist}}
\newenvironment{exercise}[2][Exercise]{\begin{trivlist}
\item[\hskip \labelsep {\bfseries #1}\hskip \labelsep {\bfseries #2.}]}{\end{trivlist}}
\newenvironment{problem}[2][Problem]{\begin{trivlist}
\item[\hskip \labelsep {\bfseries #1}\hskip \labelsep {\bfseries #2.}]}{\end{trivlist}}
\newenvironment{question}[2][Question]{\begin{trivlist}
\item[\hskip \labelsep {\bfseries #1}\hskip \labelsep {\bfseries #2.}]}{\end{trivlist}}
\newenvironment{corollary}[2][Corollary]{\begin{trivlist}
\item[\hskip \labelsep {\bfseries #1}\hskip \labelsep {\bfseries #2.}]}{\end{trivlist}}
\newenvironment{solution}{\begin{proof}[Solution]}{\end{proof}}
\newenvironment{idea}[2][Proof Idea.]{\textit{#1} #2}



\parindent0in
\pagestyle{plain}
\thispagestyle{plain}

\usepackage{csquotes}
\usepackage[shortlabels]{enumitem}

\newcommand{\dated}{\today}
\newcommand{\token}[1]{\langle \text{#1} \rangle}

\begin{document}

\textbf{Introduction to the Theory of
Computation}\hfill\textbf{\myname}\\[0.01in]
\textbf{Chapter 5: Reducibility}\hfill\textbf{\dated}\\
\smallskip\hrule\bigskip

\begin{problem}{5.26}
Define a \textbf{\textit{two-headed finite automaton}} (2DFA) to be a deterministic finite automaton that has two read-only, bidirectional heads that start at the left-hand end of the input tape and can be independently controlled to move in either direction. The tape of a 2DFA is finite and is just large enough to contain the input plus two additional blank tape cells, one on the left-hand end and one on the right-hand end, that serve as delimiters. A 2DFA accepts its input by entering a special accept state. For example, a 2DFA can recognize the language ${a^nb^nc^n \ | \ n \geq 0}$.
\end{problem}

\begin{problem}[Part]{a}
Let $A_{2DFA} = \{\langle M, x \rangle \ | \ M \text{ is a 2DFA and } M \text{ accepts } x\}$. Show that $A_{2DFA}$ is decidable.
\end{problem}

\begin{proof}
First, we define a notation for representing different configurations of a 2DFA, and then we calculate the number of distinct configurations of a 2DFA for an input of length $n$. \\

For a state $q$ and two strings $u$ and $v$ over the input alphabet $\Sigma$, we write $q \sqcup \circ \ u \bullet v \ \sqcup$ for the configuration  where the current state is $q$, the current input is $uv$, the current first and second head locations are the first symbols of $u$ and $v$ respectively. For a 2DFA with $q$ states, there are exactly $q(n + 2)^2$ distinct configurations for an input of length $n$. Construct decider $S$ for $A_{2DFA}$ as follows. \\

$S =$ \textquotedblleft On input $\langle M, x \rangle$, where $M$ is a 2DFA and $x$ is a string:
\begin{enumerate}
\item Let $n$ be the length of string $x$.
\item Simulate $M$ on $x$ for $q(n + 2)^2$ steps or until it halts.
\item If $M$ has halted, \textit{accept} if it has accepted and \textit{reject} if it has rejected. If it has not halted, \textit{reject}.\textquotedblright
\end{enumerate}
\end{proof}

\begin{problem}[Part]{a}
Let $E_{2DFA} = \{\langle M \rangle \ | \ M \text{ is a 2DFA and } L(M) = \emptyset   \}$. Show that $E_{2DFA}$ is not decidable.
\end{problem}

\begin{idea}
The proof is by reduction from $A_{TM}$. We show that if $E_{2DFA}$
were decidable, $A_{TM}$ would also be. For a \textbf{TM} $M$ and an input $w$, we can determine whether $M$ accepts $w$ by constructing a certain 2DFA $B$, such that the language that $B$ recognizes comprises all accepting computation histories for $M$ on $w$. \\

We construct $B$ to accept its input $x$ if $x$ is an accepting computation history $C_1,C_2,\dots,C_l$ for $M$ on $w$. We assume that the accepting computation history is presented as a single string with the configurations separated from each other by the \# symbol, such as $\#C_1\#C_2\#\dots\#C_l\#$. Given and input $x$, $B$ determines
whether the $C_i\text{'s}$ satisfy the three conditions of an accepting computation history.
\begin{enumerate}
\item $C_1$ is the start configuration for $M$ on $w$.
\item Each $C_{i+1}$ legally follows from $C_i$.
\item $C_l$ is an accepting configuration for $M$.
\end{enumerate}
The start configuration $C_1$ for $M$ on $w$ is the string $q_0w_1w_2 \dots w_n$, where $q_0$ is the start state for M on w. $B$ has this string directly built in, so it is able to check the first condition. An accepting configuration is one that contains the $q_{accept}$ state, so $B$ can check the third condition by scanning $C_l$ for $q_{accept}$. To make sure that the second condition is satisfied, $B$ checks on whether $C_{i+1}$ legally follows from $C_i$. This step involves verifying that $C_i$ and $C_{i+1}$ are identical except for the positions under and adjacent to the head in $C_i$. These positions must be updated according to the transition function of $M$.
\end{idea}

\begin{proof}
Assume that \textbf{TM} $R$ decides $E_{2DFA}$. Construct \textbf{TM} $S$ to decide $A_{TM}$ as follows. \\

$S =$ \textquotedblleft On input $\langle M, w \rangle$, where $M$ is a \textbf{TM} and $w$ is a string:
\begin{enumerate}
\item Construct 2DFA $B$ from $M$ and $w$ as described in the proof idea.
\item Run $R$ on $\langle B \rangle$.
\item If $R$ rejects, $M$ accepts $w$, so \textit{accept}. Otherwise, \textit{reject}.\textquotedblright
\end{enumerate}
Thus, if \textbf{TM} $R$ exists, we can decide $A_{TM}$, but we know that $A_{TM}$ is undecidable\footnote{Theorem 4.11 $A_{TM}$ is undecidable.}. By virtue of this contradiction, we can conclude that $R$ does not exist. Therefore, $E_{2DFA}$ is undecidable.
\end{proof}

\end{document}
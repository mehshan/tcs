\documentclass[11pt]{article}

\usepackage{fullpage}
\usepackage{graphicx}
\usepackage{amsmath}
\usepackage{amssymb}
\usepackage{amsthm}
\usepackage{fancyvrb}

\parindent0in
\pagestyle{plain}
\thispagestyle{plain}

\newcommand{\myname}{Mehshan Mustafa}
\newcommand{\dated}{\today}

\newenvironment{theorem}[2][Theorem]{\begin{trivlist}
\item[\hskip \labelsep {\bfseries #1}\hskip \labelsep {\bfseries #2.}]}{\end{trivlist}}
\newenvironment{lemma}[2][Lemma]{\begin{trivlist}
\item[\hskip \labelsep {\bfseries #1}\hskip \labelsep {\bfseries #2.}]}{\end{trivlist}}
\newenvironment{exercise}[2][Exercise]{\begin{trivlist}
\item[\hskip \labelsep {\bfseries #1}\hskip \labelsep {\bfseries #2.}]}{\end{trivlist}}
\newenvironment{problem}[2][Problem]{\begin{trivlist}
\item[\hskip \labelsep {\bfseries #1}\hskip \labelsep {\bfseries #2.}]}{\end{trivlist}}
\newenvironment{question}[2][Question]{\begin{trivlist}
\item[\hskip \labelsep {\bfseries #1}\hskip \labelsep {\bfseries #2.}]}{\end{trivlist}}
\newenvironment{corollary}[2][Corollary]{\begin{trivlist}
\item[\hskip \labelsep {\bfseries #1}\hskip \labelsep {\bfseries #2.}]}{\end{trivlist}}
\newenvironment{solution}{\begin{proof}[Solution]}{\end{proof}}
\newenvironment{idea}[2][Proof Idea.]{\textit{#1} #2}

\begin{document}

\textbf{Introduction to the Theory of
Computation}\hfill\textbf{\myname}\\[0.01in]
\textbf{Chapter 1: Reqular Languages}\hfill\textbf{\dated}\\
\smallskip\hrule\bigskip

\begin{problem}{1.51}
Let $x$ and $y$ be strings and let $L$ be any language. We say that $x$ and $y$ are \textbf{distinguishable by $L$} if some string $z$ exists whereby exactly one of the strings $xz$ and $yz$ is a member of $L$; otherwise, for every string $z$, we have $xz \in L$ whenever $yz \in L$ and we say that $x$ and $y$ are \textbf{indistinguishable by $L$}. If $x$ and $y$ are indistinguishable by $L$, we write $x \equiv_{L} y$. Show that $\equiv_{L}$ is an equivalence relation.
\end{problem}

\begin{proof}
To show that $\equiv_{L}$ is an equivalence relation, first give formal definition of $\equiv_{L}$\footnote{The formal method used in this proof is presented in the book \textbf{A Logical Approach
to Discrete Math} by David Gries and Fred B. Schneider. I recommend the course \textbf{Math 220: Formal Methods, Pepperdine University} to any who wishes to learn more about this method. Course link is https://cslab.pepperdine.edu/warford/math220/.}, and then show that $\equiv_{L}$ is:
\begin{enumerate}
\item Reflexive
\item Symmetric
\item Tranisitive
\end{enumerate}
\begin{theorem}[Definition]{1} 
Let $x$ and $y$ be strings and let $L$ be any language. Then $x \equiv_{L} y$ is defined as
\[
(x \equiv_{L} y) \equiv (\forall z \ |: xz \in L \equiv yz \in L).
\]
\end{theorem}

\begin{problem}[Part]{1}
Reflexivity: $x \equiv_{L} y$
\end{problem}

\begin{problem}[Part]{2}
Symmetry: $x \equiv_{L} y \Rightarrow y \equiv_{L} x$
\end{problem}

\begin{problem}[Part]{3}
Transitivity: $x \equiv_{L} y \wedge y \equiv_{L} w \Rightarrow x \equiv_{L} w$
\end{problem}
\end{proof}
\end{document}